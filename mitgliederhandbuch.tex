% RZL-Mitgliederhandbuch
% © 2012 Michael Stapelberg
\documentclass[12pt, DIV16, a4paper]{scrartcl}
\usepackage[utf8]{inputenc}

\usepackage{fancyhdr}
\usepackage{avant}
\usepackage{cmbright} % elegante serifenlose Schrift im Helvetica-Stil

\usepackage{listings}
\usepackage{cmap}
\usepackage{hyphenat}
\usepackage{graphicx}
\usepackage{fixltx2e}
\usepackage{microtype}

%\usepackage[ae]{babel}
%\usepackage[square]{natbib}
\usepackage[pdftex,bookmarks=true,bookmarksnumbered=true,bookmarksopen=true,colorlinks=true,filecolor=black,
                linkcolor=red,urlcolor=blue,plainpages=false,pdfpagelabels,citecolor=black,
                pdftitle={doctitle},pdfauthor={Michael Stapelberg}]{hyperref}
\lstnewenvironment{code}{%
        \lstset{frame=single, language=SQL,%
                showstringspaces=false, basicstyle=\footnotesize\ttfamily}
}{}

\begin{document}
\pagestyle{fancy}
\lhead{RZL}
\chead{Mitgliederhandbuch}
\rhead{2017-01-18}
\newcommand{\np}{\bigskip\noindent}
\setlength{\parindent}{0pt}

\section*{Einleitung}

Dieses Handbuch erklärt dir, wie du dich erfolgreich im RaumZeitLabor
einbringen kannst. Zunächst freuen wir uns, dass du Mitglied im RaumZeitLabor
(geworden) bist!
\np

In einem Hackerspace finden sich viele verschiedene Menschen ein, und das macht
einen großen Teil des Charmes aus. Zum erfolgreichen Zusammenleben sollte man
daher folgenden Grundsatz beachten:

\begin{center}
	Respektiere die anderen Mitglieder, sei freundlich und cool.
\end{center}

Alles Wissenswerte erfährst du in den folgenden Abschnitten.

\section*{Zugang zum RaumZeitLabor}

Als Mitglied erhältst du eine PIN, mit der du die Tür zum RaumZeitLabor
aufschließen kannst. Melde dich dazu -- möglichst mit deinem echten Namen, um
dem Vorstand die Zuordnung zu erleichtern -- in der
\href{https://benutzerdb.raumzeitlabor.de/BenutzerDB/register}{BenutzerDB} an. Danach
kannst Du den Vorstand bitten, dir eine PIN zuzuteilen. Hierzu benötigen sie
deinen Benutzernamen.

\section*{Wöchentliches Treffen}

Die meisten Laboranten triffst du Dienstag abends an. Jede Woche treffen wir
uns zur \textbf{Offenen RaumZeitLaborierung}. Zu dieser Veranstaltung sind
Interessierte ganz besonders eingeladen, da sie immer gut besucht und
interessant ist.
\np

Generell gilt: Sprich uns an, wenn du mit uns reden möchtest, und frag nach,
wenn dich etwas interessiert. Oftmals wirken wir vertieft in Projekte, aber wir
erzählen dir gerne darüber! Manchmal erklären wir zu schnell, oder
unverständlich. Frag bitte nach!

\section*{Plenum, Wipe \& Defrag}

Jeden ersten Sonntag im Monat halten wir ein Plenum. Das Plenum ist ein
öffentliches Treffen von Mitgliedern (und denen die es werden wollen), auf dem
wir aktuelle Themen besprechen und kleinere Entscheidungen treffen können. Die
zu besprechenden Punkte sammeln wir im Wiki auf der Seite
\href{https://wiki.raumzeitlabor.de/wiki/Plenum}{``Plenum''}. Wenn Du Ideen hast, wie
wir das Zusammenleben im Verein verbessern können, oder dir Dinge aufgefallen
sind, die dir nicht gepasst haben, trage diese einfach auf der Seite ein. Auf
dem Plenum ist dann die Gelegenheit, in Ruhe darüber zu reden, und
Verbesserungen vorzuschlagen. Auch entscheiden wir auf dem Plenum über kleinere
Anschaffungen. \np

``Wo gehobelt wird fallen Späne'' -- so natürlich auch bei uns. Deswegen treffen
wir uns \emph{vor} dem Plenum, um den Raum auf Vordermann zu bringen
(\href{https://wiki.raumzeitlabor.de/wiki/Wipe_\%26_Defrag}{``Wipe \& Defrag''});
üblicherweise gegen 14 bis 15 Uhr. Jeder schnappt sich einen Besen, Kehrblech,
Staubsauger oder Wischmob und gibt sein Bestes. Natürlich möchten wir niemanden
zwingen, zum Putzen auftauchen -- wenn Du allerdings häufig da bist, würden wir
uns wünschen, dass Du uns auch bei diesen eher unliebsamen Aufgaben unterstützt.
Das Ganze hat übrigens auch Vorteile: erstens gibt es kostenlose Verpflegung in
Form von Pizza, und zweitens bekommst Du mit, wann das Plenum beginnt -- den
genauen Termin kündigen wir nämlich üblicherweise nicht an.

\section*{Internet}

Das RaumZeitLabor besitzt einen eigenen Internetanschluss. Dieser steht jedem
zur Verfügung -- entweder per WLAN (den Key bekommst Du auf Anfrage), oder per
LAN über unsere Netzwerkssteckdosen, welche überall im Raum verteilt sind. Bitte
halte im Hinterkopf, dass Du dir den Internetanschluss mit anderen Laboranten
teilst und benutze ihn dementsprechend rücksichtsvoll.
\np

Da es in Deutschland leider immer noch das bisher rechtlich ungelöste Problem
der Störerhaftung gibt, tunneln wir den Traffic derzeit über einen Endpunkt nach
draußen. Vergiss nicht, dass die hierfür verantwortliche Person seinen Kopf für
uns hinhält. \textbf{Aus diesem und anderen Gründen möchten wir daher nicht,
dass du den Internetanschluss für P2P benutzt!}

\section*{Mailingliste}

Es gibt eine Mailingliste für das RaumZeitLabor, welche Interessierte abonniert
haben:\\
\url{https://lists.raumzeitlabor.de/mailman/private/raumzeitlabor/}
\np

Auf dieser Mailingliste findest du oftmals Ankündigungen und generelle
Mitteilungen zu allen möglichen Themen. Auch kann es schon einmal hoch her
gehen; sei aber unbesorgt, in der Regel verstehen wir uns alle sehr gut.
\np

Zusätzlich zur ``Haupt''-Mailingliste gibt es auch noch ``info''. Diese wird als
Kontaktadresse auf unseren Seiten verwendet, damit Anfragen möglichst schnell
von den entsprechenden Leuten beantwortet werden können (möglicherweise also
auch von dir!).

\section*{Wiki}

In unserem Wiki findet sich Dokumentation zu allem, was sich im RaumZeitLabor
an Equipment befindet (wenn nicht, ergänze es!):
\url{https://wiki.raumzeitlabor.de/}

\section*{Internet Relay Chat (IRC)}

Viele Mitglieder finden sich im Internet Relay Chat (IRC) zum lockeren
Quatschen. Verbinde dich mit dem Server \texttt{irc.hackint.net} und leiste uns
im Channel \texttt{\#raumzeitlabor} Gesellschaft. Benutze am besten SSL.
\np

Im IRC kannst du herausfinden, ob das RaumZeitLabor gerade offen ist, indem du
den Befehl \texttt{!raum} schickst. Du kannst die Rundumleuchte im Raum
aktivieren um die Anwesenden auf deine Nachrichten im IRC hinzuweisen, indem du
\texttt{!ping} schickst. Wenn Du hinter \texttt{!ping} noch eine Nachricht
hängst, so wird diese auf dem LCD-Display über der Tafel im Raum angezeigt.
\np

Zusätzlich gibt es die Möglichkeit herauszufinden, welche Laboranten gerade im
Raum sind. Alle anwesenden Laboranten haben Voice (\texttt{+v}). Voraussetzung
hierfür ist, dass das jeweilige Mitglied die MAC-Adressen seiner Geräte in der
BenutzerDB eingetragen hat (und das ist natürlich freiwillig).

\section*{Sourcecode und Projektdateien: Github}

Für Versionskontrolle verwenden wir überwiegend
git\footnote{\url{https://www.git-scm.com/}} und hosten unsere git-repositories
auf \href{https://github.com/}{github} unter der Adresse
\url{https://github.com/raumzeitlabor}. Du kannst ein neues Repository anlegen,
indem du dich an einen der Administratoren auf github wendest, derzeit sind das
u.a.\ tiefpunkt, Felicitus und Else.

\section*{Das ist kaputt! / Ich will was machen! / Unser Space soll schöner
werden!}
Falls dir etwas auffällt, das Reperatur, Verbesserung oder Pflege benötigt
findest du unter \url{https://github.com/raumzeitlabor/rzl-tuwat/issues} eine
Liste bekannter Issues und die Möglichkeit neue zu eröffnen.
Du bist herzlich eingeladen dich an den
Diskussionen und vor allem an der Lösung der Probleme zu beteiligen. Ein Blick
in die "closed issues" lohnt ebenfalls, denn möglicherweise gab es das Problem
schon einmal und du kannst auf der Diskussion / dem Wissen anderer
Laborantinnen und Laboranten aufbauen.

\section*{Vorträge}

Wir laden dich herzlich dazu ein, einen Vortrag zu halten. Ein wichtiges
Konzept im RaumZeitLabor ist Lernen durch Lehren. Ganz egal wozu du etwas
erzählen möchtest, es interessiert mit Sicherheit irgendwen. Auch wenn du
denkst, dass dein Thema doch nicht der Rede wert sei, täuscht das in aller
Regel. Was für dich klar ist, ist vielen neu. Daher: Halte einen Vortrag!
\np

Eine gute Gelegenheit für Vorträge in lockerer Runde ist Dienstags bei der
offenen RaumZeitLaborierung. Wende dich an thinkJD (Jan-Daniel), der das
dienstägliche Programm verwaltet.

\section*{Blog}

Auf \url{https://raumzeitlabor.de/blog/} schreiben wir über alle möglichen
Neuigkeiten aus dem RaumZeitLabor. Viele Menschen lesen das Blog und daher wäre
es gut, wenn du neue Beiträge beisteuern könntest -- zum Beispiel, wenn ein
Projekt Fortschritte macht. Forke dazu das
\href{https://github.com/raumzeitlabor/rzl-homepage/}{Repositoy} des Blogs,
verfasse den Blogpost und erstelle einen Pullrequest.
\end{document}
